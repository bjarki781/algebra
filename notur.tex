\documentclass[a4paper, 14pt]{article}

\usepackage[margin=1in]{geometry}

\usepackage[skip=0.1in]{parskip}
\renewcommand{\familydefault}{\sfdefault}

\usepackage[icelandic]{babel}
 \usepackage[T1]{fontenc}

\usepackage{hyperref}
\usepackage{amsmath}
\usepackage{amssymb}
\usepackage{amsthm}
\usepackage{lmodern}
\usepackage{graphicx}
\usepackage{xcolor}
\usepackage{float}
\usepackage{mathtools}
\usepackage{wrapfig}
\usepackage{lipsum}
\usepackage{dcolumn}
\usepackage{caption}
\usepackage{subcaption}
\usepackage{appendix}
\usepackage{booktabs}
\usepackage{tabu}
\usepackage{cancel}
\usepackage{xparse}
\usepackage{tikz-cd}

\usepackage{pgfplots}
\pgfplotsset{width=0.76\textwidth, height=6cm, compat=1.9}

\usepackage{xcolor}
\usepackage{listings}

\usepackage{sectsty}
\providecommand{\tightlist}{\setlength{\itemsep}{0pt}\setlength{\parskip}{0pt}}

\usepackage{csquotes}
\usepackage[backend=biber, citestyle=alphabetic, style=alphabetic]{biblatex}

\DeclareMathOperator{\atan}{atan}
\DeclareMathOperator{\Spec}{Spec}
\DeclareMathOperator{\Ann}{Ann}
\DeclareMathOperator{\V}{V}
\DeclareMathOperator{\I}{I}
\DeclareMathOperator{\D}{D}
\DeclareMathOperator{\Ass}{Ass}
\DeclareMathOperator{\Mor}{Mor}
\DeclareMathOperator{\Hom}{Hom}
\DeclareMathOperator{\Supp}{Supp}
\DeclareMathOperator{\Min}{Min}
\DeclareMathOperator{\MaxSpec}{Max\, Spec}
\DeclareMathOperator{\Quot}{Quot}
\DeclareMathOperator{\Span}{Span}
\DeclareMathOperator{\id}{id}
\DeclareMathOperator{\Ker}{Ker}
\DeclareMathOperator{\Cone}{Cone}

\newtheorem{theorem}{Setning}
\newtheorem{proposition}{Fullyrðing}
\newtheorem{example}{Dæmi}
\newtheorem{corollary}{Fylgisetning}
\newtheorem{lemma}{Lemma}

\let\surfun\twoheadrightarrow
\let\infun\hookrightarrow
\newcommand{\ru}{\noindent\rule{\textwidth}{0.4pt}}
\let\tensor\otimes
\let\dsum\oplus
\NewDocumentCommand{\xs}{O{x} O{n}}{#1_1, \dots, #1_#2}
\newcommand{\vx}{\boldsymbol{x}}
\newcommand{\C}{\mathbb{C}}
\newcommand{\R}{\mathbb{R}}
\newcommand{\Z}{\mathbb{Z}}
\newcommand{\N}{\mathbb{N}}
\newcommand{\A}{\mathbb{A}}
\newcommand{\then}{\Longrightarrow}
\newcommand{\ergo}{$\Longrightarrow$}
\newcommand{\m}{\mathfrak{m}}
\renewcommand{\P}{\mathcal{P}}
\renewcommand{\Im}{\operatorname{Im}}


\DeclarePairedDelimiter{\nint}{\lfloor}{\rceil}

\binoppenalty=\maxdimen
\relpenalty=\maxdimen


\usepackage{hyperref}
\usepackage{subfiles}

\title{Algebra I nótur}
\date{Vetrarönn 2022}

\begin{document}
\maketitle


\section{Algebra}

Látum $R$ vera baug og $I \leq R$ vera íðal. Rótaríðal $I$, táknað $\sqrt{I}$, er íðal staka sem lenda í $I$ þegar þau eru sett í nógu hátt veldi.
$$
    \sqrt{I} = \{ r \in R \mid r^n \in I, n \in \N \}.
$$
Einn eiginleiki rótaríðala er að þau eru sniðmengi lágíðala $R/I$.
\[
    \sqrt{I} = \bigcap \Min(R/I) = \bigcap \Ass(R/I) ?
\]
og sértilfelli af þessu er
\[
    \sqrt{0} = \bigcap \Min(R) = \bigcap \Spec(R)
\]
Athugið að $\sqrt{0}$ er mengi núllvalda staka $R$.

\ru

Látum $M$ vera $R$-mótul. Eyðir (e. annihilator) $S$ er íðal í $R$ hvers stök eru núlldeilar staka $S$. Eyðir $S$ er táknaður $\Ann_R(S) = \{ r \in R \mid s \in S \then rs = 0 \}$. Athugum að $0 \in \Ann_R(S)$ fyrir öll $S$. Tengt frumíðöl $M$ í $R$ eru frumíðöl í $R$ sem 
eyða hlutmótlum í $M$. Mengi tengdra frumíðala er táknað $\Ass_R(M) = \{ P \in \Spec(R) \mid N \leq M \then PN = 0 \}.$
Það er líka hægt að segja þetta fram með 
$P \in \Ass_R(M) \iff \exists R/P \infun M$ eða $\exists m \in M, \Ann_R(m) = P$.

\ru

Núlldeilir í baugnum $R$ er stak $r \in R \setminus \{ 0 \} $ þ.a. $\exists t \in R \setminus \{ 0 \}$ svo $rt = 0$.
Eins tölum við um núlldeila fyrir $R$-mótulinn sem stak $r \in R \setminus \{ 0 \} $ þ.a. $\exists m \in M \setminus \{ 0 \}$ svo $rm = 0$.

$$
    \{ \text{núlldeilar í } M \} \cup \{ 0 \} = \bigcup \Ass(M)
$$

$$
    \sqrt{0_R} = \{ \text{núllvalda stök } R \} = \bigcap \Spec(R) = \bigcap \Min(R) = \bigcap \Ass(R)
$$

Einfaldur mótull eða höfuðmótull (e. simple module) er mótull sem á enga aðra hlutmótla fyrir utan sig og núllmótulinn.
Önnur skilgreining á höfuðmótlum er að það er mótull  sem er spannaður af hverju ekki-núll staki $M^*$ sínu.
Höfuðmótlar eru því spannaðir af einu staki, líkt og höfuðíðöl.
Þá er til mótun $R \surfun M$. Ef við
deilum með kjarna þessarar mótunar fáum við $R/I \xrightarrow{\sim} M$. Höfuðmótlar er alltaf einsmóta $R/I$ þar sem $I$ er kjarni varpanarinnar. Ef deilibaugurinn $R/I$ er svið þá $I \in \MaxSpec(R)$. Þar sem $M$ á sér aðeins einn
hlutmótul, nefnilega $(0)$, þá á $R/I$ sér aðeins eitt íðal, $(0)$, og er þá svið

Tökum þetta saman og fáum að $M$ er höfuðmótull ef og aðeins ef $M \cong R/\mathfrak{m}$ þar sem $\mathfrak{m} \in \MaxSpec(R)$.

Háíðöl í $\C[x]$ eru alltaf á forminu $\mathfrak{m}_c = (x-c)$, þ.e. punktar.

\subsection{Turnar/síarnir}
Turn af hlutmótlum í $R$-mótlinum $M$ er endanleg fallandi runa $M_0 \supset M_1 \supset \cdots \supset M_k$. 
Þægir turnar (e. nice filtrations) eru turnar sem hafa deildamótla á forminu $M_i/M_{i+1} \cong R/P_i$. Eftirfarandi turn er þægur:
$$
    M_0 = M = \C / (x^3, y^2) \supset \cdots \supset \C / (y) \supset 0 = M_k.
$$
Ef $R$ er noetherískur og $M$ er endanlega spannaður þá er alltaf til þægur turn í $M$.

\ru

Turn kallast \emph{Jordan-Hölder turn} (e. \emph{composition series}) ef allir deildarmótlar aðliggjandi hlutmótla
eru einfaldir mótlar.

Lengd stysta Jordan-Hölder-turnsins í $M$ táknum við $\ell(M)$. $\ell(M) \in \N \cup \{ \infty \}$.
Nokkrir eiginleikar $\ell(m)$ 
\begin{enumerate}
  \item $\ell(M)$ er strangt stækkandi (e. \emph{strictly monotonically increasing}). Það er,
  ef $N \subset M$ þá  $\ell(N) < \ell(M)$.
  \item Lengd hverrar síunnar í $M$ getur mest verið $\ell(M)$
  \item Ef $0 \to M' \to M \to M'' \to 0$ er fleyguð þá $\ell(M) = \ell(M') + \ell(M'')$.
  \item Síun er Jordan-Hölder-turn þþaa lengd síunnarinnar er $\ell(M)$.
\end{enumerate}
R-mótull $M$ er bæði artinískur og noetherískur þá og því aðeins að $\ell(M) < \infty$.

\subsection{Noetherískir og artinískir mótlar}
Mótullinn $M$ er noetherískur, þ.e. Noether-mótull þ.þ.a.a.\ hver stækkandi runa hlutmótla hans stansar.
Þetta er kallað stækkandi runuskilyrðið (e. ascending chain condition, ACC).
Mótullinn $M$ er artinískur, þ.e. Artin-mótull þ.þ.a.a. hver minnkandi runa hlutmótla hans stansar. 
Þetta er kallað minnkandi runuskilyrðið (e. descending chain condition, DCC).
Hugtökin artinísk og noetherísk eru samhverf fyrir mótla (ekki fyrir bauga).
Dæmi um noetherískan $\Z$-mótul er $Z$. 

Ef $M$ er noetherískur þá eru allir hlutmótlar $M$ eru endanlega spannaðir. Einnig er $M$ noetherískur 
þ.þ.a.a.\ hvert $K \leq M$ og $M/K$ eru noetherískir. og allir hlutmótlar allra endanlega spannaðra $R$-mótla eru 
einning endanlega spannaðir.

\subsection{Noetherískir og artinískir baugar}
Nokkrir eiginleikar artnískra og noetherískra bauga.
\begin{itemize}
\item Baugur er noetherískur/artinískur ef hann er noetherískur/artinískur
sem mótull. 
\item (Hilbert's basis theorem) Ef $R$ er noetherískur þá er $R[x]$ noetherískur (og þ.a.l einnig $R[x_i, \dots, x_n]$).
$R$ er Artin-baugur $\iff \ell(R) < \infty \iff \MaxSpec(R) = \Spec(R)$ og $R$ er noetherískur ($|\Spec(R)| < \infty$).

\item $\MaxSpec(R) = \Spec(R) \iff \dim(R) = 0 \; $(R er þá svið).

\item $\dim(R) = \sup \{ \ell(P) \mid P \in \Spec(R) \}$.
Athugið að baugur $R$ getur ekki átt sér vídd líkt og vigursvið en hér merkir
$\dim(R)$ \emph{Krull-vídd} $R$.

\item $\ell_R(R) < \infty \then$ DCC og ACC.

\item $P \in \Supp(R/I)$ þ.þ.a.a.\ $P \supset I$, $P$ er frumíðal.

\item Öll íðöl $I \leq R$ noetherískur er hægt að skrifa sem endanlegt sniðmengi frumíðala.
$I = P_1 \cap \cdots \cap P_n$. Fjölskyldan $\{ P_i \}_I$ þarf ekki að vera einstök.
Til dæmis $I = (x^2, xy)$. $I = (x, y^2) = (x) \cap (x^2, xy, y^2).$
SKILJA DÆMIÐ!!!

\end{itemize}

Íðal $I = Q_1 \cap \dots \cap Q_n$ kallast óþættanlegt ef og aðeins ef 
$I \subset Q_1 \cap \dots \cap \, \cancel{Q_i} \, \cap \cdots Q_n$ og $i \neq j \then \sqrt{Q_i} \neq \sqrt{Q_j}$.

Íðal $I = Q_1 \cap \dots \cap Q_n$ er endanleg þáttun. Þá $\{ \sqrt{Q_1}, \dots, \sqrt{Q_n} \} = \Ass(R/I).$

$\Ass(R/I) \subseteq \Min(R/I)$.
Ef $I = Q_1 \cap \dots \cap Q_n$ og $\sqrt{Q_i} \in \Min(R/I)$ þá
er þátturinn $Q_i$ einstakt ákvarðaður (hann er alltaf þáttur í öllum
þáttunum á $I$).

Baugurinn $R$ er noetherískur þá og því aðeins að allir endanlega spannaðir $R$-mótlar eru noetherískir.

$R$ noetherískur $\then$ Öll íðöl $R$ eru endanlega spönnuð.
\begin{itemize}

    \item $\then$  $R/I$ er noetherískur
    \item $\then$  $S^{-1}R$ er noetherískur (sér í lagi $R_P$ og $R_f$, og þ.a.l. $M_P$ og $M_f$).

\end{itemize}

\subsection{Lágfrumíðöl}
Lágfrumíðal $R$ er íðal $R$ sem er hlutmengi allra íðala sem skarast á við það. Við táknum
lágfrumíðul baugsins $R$ sem $\Min(R)$. Þetta getum við skrifað með táknum
\[
    \Min(R) = \{ P \in \Spec(R) \mid  \nexists Q \in \Spec(R), 0 \subset Q \subset P \}.
\]
Ef við skoðum sérstaklega hvernig deilibaugar haga sér höfum við eftirfarandi jöfnu
\[
    \Min(R/I) = \{ P \in \Spec(R) \mid \nexists Q \in \Spec(R), P \supset Q \supset I.
\]

\subsection{Staðbinding}
Eins og nafnið gefur til kynna er staðbinding (e. localization) leið til þess að staðbinda mótla.
Staðbundinn mótull $M$ á sér aðeins eitt einstakt háíðal $m$ og hann er oft tengur háíðalinu sínu
í rithætti og skrifaður $(M, m)$. Ef við tökum baug $R$ og frumíðal $P \leq R$ getum við staðbundið
$R$ sem mótul. Þ.e. $R$ og $P$ verður $(R_P, PR_P)$.
Deildarbaugur baugs og háíðals hans er svið. Við táknum þetta svið sem staðbindingin m.t.t.\ $P$ og háíðal þess mynda
\[
    K(P) : = \Quot R/P = R_P/PR_P
\]

$$
\Supp(M) = \{ P \in \Spec(R) \mid M_P \neq 0 \} 
$$
Staðbinding er fleygaður varpi (e. exact functor).

\subsection{Frymin íðöl (e. primary ideals)}
Baugurinn $R$ er noetherískur. $Q \leq R$, $Q$ er frymið $\iff \sqrt{0_{R/Q}} = \text{núllvalda stök } R/Q$.
Aukasetning: $Q$ er frumíðal $\then$ $Q$ er frymið.

$ab \in Q \then a,b \in \sqrt{Q}$

$Q$ er frymið $\then$ $\sqrt{Q}$ er frumíðal.

$Q$ er frymið og $P = \sqrt{Q} \then $ $Q$ kallast $P$-frymið

\begin{lemma}
$\m \in \MaxSpec(R), \m = \sqrt{Q} \then Q$ er frymið.
\end{lemma}
\begin{proof}
Skoðum $Q$ í $R/Q$. Þá er $Q = \{ 0 \}$ og $\sqrt{Q} = \sqrt{0}$. Við vitum
að við getum skrifað mengi núllvalda staka baugs sem sniðmengi allra frumíðala baugsins.
Þar sem $\m$ er háíðal sjáum við að það verður að vera jafn yfiríðölum sínum, þ.e. jafnt
öllum frumíðölunum. Mengi núlldeila er sammengið $\bigcup \Ass(R/Q)$, $\Ass(R/Q)$ verandi 
frumíðöl jöfn $\m$ sjáum við að mengi núlldeila er einnig sama og $\m$. Þar með höfum við
fyrir bauginn $R/Q$ að mengi núlldeila og núllvalda staka er eitt og hið sama.

\emph{Mætti passa upp á munin á íðölum í deiliíðalinu og R}

\end{proof}

Öll $\cap$-óþáttanleg íðöl eru frymin.

\subsection{Þinmargfeldi}
Þinmargfeldi tveggja strúktura er strúktúr sem einangrar þann eiginleika sem
tvíliða föll úr þessum tveimur strúkturum eiga.
Þinmargfelgi tvegga mótla er best að skilgreina sem með allsherjareiginleikanum. 

\begin{center}
\begin{tikzcd}
        & M \tensor N \arrow[dd, dotted] \\
    M \times N \arrow[ru] \arrow[rd]  &   \\
        & P \\     
\end{tikzcd}
\end{center}

Hér eru nokkrar reiknireglur.
\[
    R \tensor_R M = M
\]
\[
    R^m \tensor_R S = S^m
\]
\[
    R/I \tensor_R R/J = R/(I + J)
\]
\[
    R/I \tensor_R M = M/MI
\]
Ef $M$ er flatur þá
\[
M \tensor_R I = MI.
\]
\[
    M/IM \tensor_{R/I} N/IN = M \tensor_R N \tensor_R R/I
\]

Ef mótull $M$ er flatur mótull þá er $ \tensor_R M$ fleygaður varpi.

\section{Heillegar baugsvíkkarnir}
Látum $A \leq B$ vera bauga. Köllum stak $b \in B$ heillegt yfir $A$ ef og aðeins ef til er
fjölskylda staka úr $A$, $\{ a_i \}_N, a_0 = 1$ sem uppfyllir eftirfarandi jöfnu
\[
    b^n a_0 + b^{n-1} a_1 + \cdots + ba_{n-1} + a_n = 0.
\]
M.ö.o. er $b$ rót fyrir margliðu (í einni breytu, $x$) með stuðlum úr $A$.
\begin{itemize}
\item $b$ er heilleg yfir $A$.
\item $A \subseteq A[b] \subseteq B$. Hér er $A[b]$ endanlega spönnuð $A$-algebra sem er einnig endanlega spönnuð sem $A$-mótull (sterkara en að vera bara e.s. $A$-algebra).  
\item $\exists$ endanleg $A$-algebra $C$: $A \subseteq A[b] \subseteq C \subseteq B$.
\item $\exists$ (e.s.) $A[b]$-mótull $M$ sem er e.s. sem $A$-mótull og $\Ann_{A[b]}(M) = 0$.
\end{itemize}

Látum $f: A \to B$ vera baugsmótun. Þessi mótun gefur okkur skilgreiningu á $B$ sem $A$-algebru.
Við segjum að $f$ sé \emph{endanleg} eff $B$ er heilleg $A$-algebra, þ.e. $\forall b \in B$, $b$ er heillegt yfir $f(A)$. 
Tvö áhugaverð tilfelli.
\begin{enumerate}
\item $f: A \twoheadrightarrow B$: $f$ er sjálfkrafa endanleg þar sem $b$ er alltaf heillegt yfir $B$.
\item $f: A \hookrightarrow B$: aftur endanleg því þá er $F=\Spec(f) : \Spec(B) \to \Spec(A)$ átæk (Going-up theorem).
    $F: Q \mapsto f^{-1}(Q)$ 
\end{enumerate}

\begin{theorem}
Lát $A$ vera heilbaug og $\Quot A \hookrightarrow L = \text{endanleg sviðsvíkkun}$.
Lát einnig $B = \overline{A}^{(L)} = \{ l \in L \mid l \text{ er heillegt yfir } A \}$.
Þá er $B$ e.s. $A$-algebra, þar af leiðandi er $B$ e.s. $A$-mótull, það er $A \to B$ er endanleg.
\end{theorem}

\begin{theorem}[Noether stöðlun I]
 Lát $k$ vera svið og $A$ vera e.s. $k$-algebru, það er $\exists k[\xs] \twoheadrightarrow A.$
Þá er til línuleg samantekt $\{ y_i \}$, það er $y_i \in \Span_k(\xs[x][d]), d \leq n$ þannig að 
\begin{enumerate}
\item $\xs[y][d]$ eru línulega sjálfvalda (?).
\item  $k[\xs] \supseteq k[\xs[y][d]] \overset{\phi}{\twoheadrightarrow} A$, $\phi$ er eintæk og endanleg (?).
\end{enumerate}
\end{theorem}

\subsection{Núllstöðusetning Hilberts}

\subsubsection{Svala útgáfan}
\begin{theorem}[Undirbúningssetning Weierstraß]
Látum $k$ vera óendanlegt svið. $f \in x[\xs]$. Það er til línuleg breyting í hnitum 
$\psi: x_i \mapsto x_i + a_ix_n$ með $\psi(f) = (\text{fasti} \neq 0) \cdot x_n^N + \sum_{i=0}^{N-1} c_i (\xs[x][n-1])
\cdot x_n^i$.
\end{theorem}

\begin{theorem}[Noether stöðlun]
Látum $k$ vera óendanlegt svið. Ef $k[\xs] \surfun A$ er endanlega spönnuð $k$-algebra þá er til $\xs[y][d]
\in \Span(\xs)$ þannig að $k[{\xs[y][d]}] \infun A$ er heilleg $k$-algebra.
\end{theorem}

\begin{corollary}[Svala útgafa núllstöðusetningar Hilberts]
Látum $k$ vera svið og $A$ vera endanlega spannaða $k$-algebra sem er líka svið. Þá er $A|k$ endanleg sviðsútvíkkun, 
þ.e. $[A : k] < \infty$.
\end{corollary}

\subsubsection{Venjulega útgáfan}
\begin{enumerate}
\item $k$ er svið, $k \hookrightarrow A$ er e.s. $k$-algebra, $A$ er svið $\then$ $k \hookrightarrow A$ er endanleg
þ.e. $A$ er endanleg sviðsvíkkun $k$. Ef $k = \overline{k}$ þá $A = k$.
\item Lát $k = \overline{k}.$ Ef $\m \in k[\xs]$ er háíðal \ergo $\exists c = (c_1, 
\dots, c_n) \in k^n: \m = \m_c := (x_1 - c_1, \dots, x_n - c_n)$.
\item $J \subseteq k[\xs], V(J) = \emptyset\then J = (1).$
\item $J \subseteq k[\xs] \then I(V(J)) = \sqrt{J}.$
\end{enumerate}

\subsection{Varpmótlar}
$R$-Mótull $P$ er varpmótull (e. projective module) ef og aðeins ef $\Hom_R(P, \bullet)$ er fleygaður varpi,
þ.e. varpi sem varðveitir fleygaðar lestir. Við vitum nú þegar að þessi varpi er alment fleygaður frá vinstri
(varðveitir eintækni) en varpmótulsskilyrðið tryggir að hann sé þá einnig fleygaður frá hægri,
þ.e. varðveitir átækni þ.e. $M \twoheadrightarrow N \then \Hom_R(P, M) \twoheadrightarrow \Hom_R(P,N).$ 

\begin{enumerate}
\item $P$ er varpmótull.
\item $\Hom_R(P, \bullet)$ er fleygaður mótull.
\item $P \twoheadrightarrow M$ klofnar, þ.e. $0 \to K \to M \twoheadrightarrow P \to 0$ klofnar, þ.e. $f$ á sér section.
\item $P$ er liður í beinni summu frjáls mótuls.
\end{enumerate}

Frjáls mótull => varpmótull => flatur mótull

$J = (2, 1 + \sqrt{5})$ er ekki frjáls mótull þar sem það er ekki höfuðíðal (frjálsir mótlar eru höfuðíðöl).
$J$ er aftur á móti varpmótull þar sem $J \dsum M = R^I $. (Finna $M$ með reiknivél)

$R$-mótull $M$ er flatur ef og aðeins ef $\bullet \tensor_R M$ er fleygaður varpi.

\section{Samsvipir og samtoganir}

Keðjufléttingum svipar mjög til fleygaðra lesta en eru aðeins almennari og þeim fylgir líka öðruvísi ritháttur.
Keðjuflétting er parið $M_\bullet=(A_\bullet, d_\bullet)$ þar sem $A_\bullet$ er runa abelískra grúpa eða mótla (eða en almennra, hlutir í 
abelísku ríki). Við skrifum $Z_i := \Ker d_i \subseteq A_i$ ($Z$ stendur fyrir \emph{Zyrkel}) og $B_i := \Im d_{i+1} \subseteq A_i$ 
($B$ stendur fyrir \emph{Boundary}). 
Einnig er sett það skilyrði á keðjufléttingar að $d_{i+1} \circ d_i = 0$, þ.e. $B_i \subseteq Z_i$ en ef við litum á 
fleygaða lest sem keðjufléttingu hefðum við að $B_i = Z_i$.

$i$-ti samsvipur keðjulestarinnar $M_\bullet$ er skilgreindur sem $H_i(M_\bullet) = Z_i / B_i$. Fyrir fleygaðar lestir 
sjáum við að $H_i = 0$ fyrir öll $i$.

Fleygaðar lestir $\subset$ Keðjufléttingar $\subset$ Lestir/Fléttingar.

\subsection{Samtogun, e. homotopy}
Í grannfræði er samtogun milli falla $f : X \to Y$ og $g : X \to Y$ samfelld tvílínuleg vörpun 
$H : [0, 1] \times X \to Y$ með  $H(0, x) = f(x)$ og $H(1, x) = g(x)$. 
Í okkar skilningi er samtogun keðjuvörpun () $h : M_\bullet \to N_\bullet[-1]$, $h_i: M_i \to M_{i+i}$
ef og aðeins ef $f_i = d^M h + h d^M = d^N h + h d^N$. Það að til sé samtogun á milli $f$ og 0 er skrifað 
$f \sim 0$.
Við fáum svo að $f \sim 0 \then H(f) = 0.$ 
$f \sim g \iff f-g \sim 0$
$f-g = hd + dh$
Lemma: $H_i(f) = H_i(g)$.

\subsection{Togjafngildi, e. homotopy equivalence}
Keðjuvörpun $f_\bullet: M_\bullet \to N_\bullet$ kallast togjafngildi ef til er $g: N_\bullet \to M_\bullet $ þannig að
$f \circ g \sim \id_{N_\bullet}$ og $g \circ f \sim \id_{M_\bullet}$ 


\subsection{Vörpunarkeila}
Látum $A_\bullet$ vera keðjufléttingu. Vörpun $f: A_\bullet \to B_\bullet$ kallast \emph{næstum einsmóta}
(e. quasi-isomorphic, qui eða qis) ef fyrir öll $i$ er $H_i(f)$ einsmótun.
Lát $f: L \to M$ vera keðjuvörpun. $\Cone f_i := L_i \oplus M_{i-1}$
$\Cone f_{i-1} := L_{i-1} \oplus M_{i-2}$

virkinn hagar sér eins og fylkjamargföldun
\[
    \begin{pmatrix}
    d & 0 \\
    f & -d_m \\
    \end{pmatrix}
    \begin{pmatrix}
    x \\ y 
    \end{pmatrix}
    =
    \begin{pmatrix}
    dx \\ fx - dmy
    \end{pmatrix}
\]

\subsection{Keðjufléttingur}
$x \in H_i = Z_i/B_i$,
$H_i(f)(x) := f(x)$.

\section{Ríkjafræði}

\subsection{Aðoka varpi}
Gefum okkur varpa $F: C \rightarrow D$, $G: D \rightarrow C$.
Ef $\Mor(FX, Y) = \Mor(X, GY)$ þá segjum við að $F$ sé aðoka (varpanum $G$)
frá vinstri og að $G$ sé aðoka (varpanum $F$) frá hægri.

Varpi $F$ sem er aðoka frá hægri er einnig fleygaður frá hægri. 
Lestin
$$
    X \rightarrow Y \rightarrow Z \rightarrow 0
$$
er fleyguð og þar af leiðandi er lestin
$$
    FX \to FY \to FZ \to 0
$$
einnig fleyguð.

\subsection{Samlagningarvarpar og abelísk ríki}
$F: C \to D$, $C$ og $D$ verandi forsamlagningargrúpur, er samlagningarvarpi ef og aðeins ef hann er grúpumótun á $\Hom$ mengjum $C$.

$F: 0 \mapsto 0$
Tenging við vörpunarkeilur

\[
    \begin{tikzcd}
    L_i \arrow{r}{f} \arrow{d}{\pi} & M_i \arrow{d}{d} \\
    L_{i-1} \arrow{r}{} & M_{i-1} \\
    \end{tikzcd}
\]

\subsection{Markgildi}
Markgildi ríkisins $C$

\section{Rúmfræði}
Við höfum áhuga á núllstöðum margliða. Við erum með formúlur fyrir sum sértilfelli, til dæmis fyrir annars stigs jöfnur 
í einni breytu. Þá er sviðið sem við vinnum í tvinntölurnar, sem eru algebrulega lokaðar. Við getum notað núllstöðvar margliða í 
fleri en einni breytu til að skilgreina alls kyns ferla, svo sem keilusniðin. Núllstöðva mengi margliðunnar 
$p(\xs) \in \C[\xs]$ er mengið 
\[
    \{ (\xs) \in \C^n \mid p(\xs) = 0 \}.
\]

Við viljum samt ekki binda okkur við tvinntölnurnar þannig við látum $k$ vera eitthvað algebrulega lokað svið,
svo sem algebrulegu tölurnar, algebruleg lokun $p$-legs sviðs, eða $\overline{\R} = \C$. Til þess að undirstrika 
rúmfræðilega eiginleika $k^n$ skrifum við það $\A_k^n$ eða $\A$ ef $k$ og víddin eru þekkt.

Tökum eftir því að margliðan $p(\xs)$ er stak í bauginum $k[\xs]$ og hún skilgreinir íðalið $J = (p(\xs))$.
Ef margliðan er óþættanleg þá skilgreinir hún frumíðal. Látum $V : \P(k[\xs]) 
\to \P(\A^n_k)$ vera fall sem tekur mengi margliða og skilar sniðmengi allra núllstöðvamengja margliðanna í $J$. 
Látum að sama skapi $I : \P(\A_k^n) \to \P(k[\xs])$ vera fall sem tekur mengi punkta og skilar öllum
margliðum sem hafa núllstöð í þessum punktum (það er íðal).


Núllstöðusetning Hilberts fjallar einmitt um tengslin þarna á milli. $I(V(J)) = \sqrt{J}$

$V(xy) = V(x) \cup V(y)$. 

$D(f) = \A / V(f)$
$D(f) = \{ p \in \Spec R \mid f \not\in p \}$

Róf (e. \emph{spectrum}) baugsins $R$ er mengi frumíðala $R$. Við skrifum venjulega róf $R$ sem $\Spec(R)$. 
\[
    V(J) = \{ P \in \Spec(R) \mid P \supset J \} = \Spec(R/J).
\]

$V(J) = V(\Ann(R/J)) = \Supp(R/J)$.

Hvað er $\Spec(k[\xs])$?
Þar sem $k$ er algebrulega lokaður eru það allar fyrsta stigs margliður í $k[\xs]$. Við höfum 
einsmótun $ \Spec(k[\xs]) \xrightarrow{\sim} \A_n^k, (x_i - c) \mapsto \pi_i(c) $.

\[
    V(f) = \{ P \in \Spec(k[\xs]) \mid P \supset (f) \} = \Spec(k[\xs]/(f)).
\]

$D(I) = \A / V(I) = \Spec(k[\xs]) / \Spec(k[\xs]/I) = \Supp I = \{ P \in \Spec \mid I_P \neq 0 \}$. $I(X)$. $\sqrt{J} = P_1 \cap \dots \cap P_n$.

\subsection{Vildarskemu (e. affine schemes)}
\[
    X = \Spec R
\]
hér er $R$ baugur fágaðra falla á $X$. (Föll keimlík margliðum.)
Gildi $r \in R$ í $P \in X$ er $\overline{r} \in K(P) := \Quot R/P = R_P/PR_P$.

\end{document}
